\paragraph{PGCD}{
	On commence par utiliser le processeur en implémentant la fonction qui
	calcule le plus grand diviseur commun de deux nombres.
	Pour ce faire, nous traduisons d'abord le code C écrit dans le sujet
	(figure \ref{pgcd_c}), puis, on traduit le code MIPS (figure 
	\ref{pgcd_asm}) en hexadécimale (figure \ref{pgcd_hexa}), en se 
	basant sur les opcodes définis plus tôt et le format des instructions.
}

\begin{figure}
	\lstset{
		frame=single,
		numbers=left,
		numbersep=5pt,
		language=C++
	}
	\begin{lstlisting}
int i = 27;
int j = 24;
while (i != j)
{
	if (i > j)
	{
		i -= j;
	}
 	else
 	{
		j -= i;
	}
}
	\end{lstlisting}
	\caption{
		\label{pgcd_c}
		Code C de la fonction \textit{pgcd} tirée du sujet
	}
\end{figure}

\begin{figure}
	\lstset{
		frame=single,
		numbers=left,
		numbersep=5pt,
		language=[mips]Assembler,
		morekeywords={halt,shr}
	}
	\begin{lstlisting}
	.data
	.text
	.globl __start
	
__start:
	li $t1, 27	#i
	li $t0, 24	#j
	
while :
	beq $t1, $t0, end_while
if:
	ble $t1, $t0, else
then:
	sub $t1, $t1, $t0
	b end_if
else:
	sub $t0, $t0, $t1
end_if:
	b while
end_while:

# fin du programme
halt

	\end{lstlisting}
	\caption{
		\label{pgcd_asm}
		Code MIPS du programme \textit{pgcd}
	}
\end{figure}

\begin{figure}
	\lstset{
		frame=single,
		numbers=left,
		numbersep=5pt,
		language=[mips]Assembler
	}
	\begin{lstlisting}
0xf11b # li $t0 24
0xf018 # li $t1 27
0x2105 # bed $t1 $t0 5
0x5102 # ble $t1 $t0 2
0x9110 # sub $t1 $t1 $t0
0x1001 # b 1
0x9001 # sub $t0 $t0 $t1
0x100a # b -6
0x0000 # halt
	\end{lstlisting}
	\caption{
		\label{pgcd_hexa}
		Code hexadécimal du programme \textit{pgcd} compilé pour \textsc{Nono 1}
	}
\end{figure}

\paragraph{}{
	Ce programme nous donne bien le résultat attendu dans les registres : $3$.
}

\paragraph{NTZ}{
	Pour aller plus loin nous avons codé un second programme. Nous avons choisi
	d'implémenter la fonction \textit{ntz} présente sur l'examen de 2012/2013 de cette
	même matière. \newline
	À la figure \ref{ntz_c} on a le code C de la fonction, à la figure \ref{ntz_asm}
	son code MIPS et à la figure \ref{ntz_asm} le code hexadécimal du programme
}

\begin{figure}
	\lstset{
		frame=single,
		numbers=left,
		numbersep=5pt,
		language=C++
	}
	\begin{lstlisting}
int ntz( unsigned int x )
{
	int n = 0;
	x = ~x & (x-1);
	while (x != 0)
	{
		n = n + 1;
		x = x >> 1;
	}
	return n;
}
	\end{lstlisting}
	\caption{
		\label{ntz_c}
		Code C de la fonction \textit{ntz} tirée du sujet de l’examen de 2012-2013
	}
\end{figure}

\begin{figure}
	\lstset{
		frame=single,
		numbers=left,
		numbersep=5pt,
		language=[mips]Assembler,
		morekeywords={halt,shr}
	}
	\begin{lstlisting}
	.data
	.text
	.globl __start
	
__start:
# initialisation des variables
	li $t0 32		# x = 32
	li $t1 0		# n = 0
	li $t3 1		# pour -1
	li $t4 0		# pour zero
		
	sub $t2 $t0 $t3		# x - 1
	not $t0 $t0		# not x
	and $t0 $t0 $t2		# and(x-1)
	
while:
	beq $t0, $t4, end_while	
	add $t1, $t1, $t3	# n = n + 1
	shr $t0, $t0, 1		# x = x >> 1
	b while			
end_while:
	# fin du programme
	halt
	\end{lstlisting}
	\caption{
		\label{ntz_asm}
		Code MIPS de la fonction \textit{ntz}
	}
\end{figure}

\begin{figure}
	\lstset{
		basicstyle=\ttfamily,
		frame=single,
		numbers=left,
		numbersep=5pt,
		language=[mips]Assembler
	}
	\begin{lstlisting}
0xf020 # li $t2 32
0xf100 # li $t1 0
0xf301 # li $t3 1
0xf400 # li $t4 0
0x9203 # sub $t2 $t0 $t3
0xc000 # not $t0 $t0
0xb002 # and $t0 $t0 $t2
0x2043 # beq $t0 $t4 3
0x8113 # add $t1 $t1 $t3
0xe003 # shr $t0 $t0 1
0x100c # b -4
0x0000 # halt
	\end{lstlisting}
	\caption{
		\label{ntz_hexa}
		Code hexadécimal de la fonction \textit{ntz}
	}
\end{figure}

\paragraph{}{
	Le programme \textit{ntz} stock son résultat dans le second registre.
}

\paragraph{HALT!}{
	D'après le fonctionnement général de \textsc{Nono 1}, l’instruction \textit{halt} donne l'adresse
	mémoire de la dernière instruction. Le problème est que le processeur ne 
	s’arrête pas lorsqu'il a atteint cette instruction. et il va boucler sur cette 
	instruction. Ainsi, si cette dernière est utilisée, elle va être exécutée à l'infini. 
}