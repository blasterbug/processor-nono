\paragraph{}{
	Un processeur \textsc{Nono 2} est un processeur \textsc{Nono 1}
	implémentant des routines. La gestion des routines suppose que
	le processeur est capable de gérer plus choses.	
}

	\subparagraph{Sauts longs}{
	Un appel à une fonction est traduit grossièrement en assembleur par 
	un saut dans le code du programme.
	Les saut longs (les instructions de type \verb|j étiquette|)
	ne sont pas les mêmes types de sauts que ceux effectué par un \verb|b|.
	Pour cela, il faut être capable d'incrémenter le pointeur d'instruction
	(\textit{PC}). Ce qui est déjà possible avec \textsc{Nono 1}.
	}
	
	\subparagraph{Paramètres de fonctions]}{
	Lorsqu'on appelle une fonction, en général, c'est pour effectuer un calcul. 
	Calcul qu'on souhaite effectuer sur des paramètres spécifique. Or il faut 
	être capable de passer ces paramètres à une fonction mais également récupérer
	son résultat, le retour de la fonction. Lorsqu'on fait appelle à une fonction
	\textit{simple} (prenant en paramètres des scalaires), il suffit d'initialiser
	des registres spécifiques 
	}