\paragraph{}{
	Un processeur \textsc{Nono 2} est un processeur \textsc{Nono 1}
	implémentant des routines. La gestion des routines suppose que
	le processeur est capable de gérer plus choses.	
}

	\subparagraph{Sauts longs}{
	Un appel à une fonction est traduit grossièrement en assembleur par 
	un saut dans le code du programme.
	Les sauts longs (les instructions de type \verb|j étiquette|)
	ne sont pas du même type que les sauts effectués par un \verb|b|.
	Pour cela, il faut être capable d'incrémenter le pointeur d'instruction
	\textit{PC}. Ce qui est déjà possible avec \textsc{Nono 1}.
	}
	
	\subparagraph{Paramètres de fonctions et pile}{
	Lorsqu'on appelle une fonction, en général, c'est pour effectuer un calcul. 
	Calcul qu'on souhaite effectuer sur des paramètres spécifiques. Il faut donc
	être capable de passer ces paramètres à une fonction mais également récupérer
	son résultat. Lorsqu'on utilise une fonction 	\textit{simple} (prenant en 
	paramètres des scalaires), il suffit d'initialiser
	des registres spécifiques pour que la fonction puisse récupérer les valeurs
	données. Puis la fonction modifie certains registres lorsqu'elle renvoie un résultat.
	Or, \textsc{Nono 1} ne permet pas cela car il n'a pas de registres dédiés. \newline
	D'autre part, lorsque la fonction prend en paramètre des types plus complexes
	(un tableau par exemple) ou qu'elle appelle elle-même une fonction, il faut
	alors gérer les cadres de pile des fonctions. Ce qui n'est pas possible actuellement.
	}
	
\paragraph{}{
	Tous ces éléments suggèrent également l'utilisation d'instructions que le processeur
	ne possède pas. Par exemple, les instructions de saut, les chargements en mémoire 
	(gestion de la pile), etc. Pour cela, il faudrait, par exemple, étendre le jeu
	d'instructions actuel. Cela entraînerai alors une modification importante du 
	processeur, notamment car la taille des opcodes ne serait plus la même. \newline
	Une seconde solution envisageable et réalisable, serait de sacrifier
	des instructions du jeu actuel au profil de nouvelles. Par exemple, une comparaison
	de type \textit{inférieur ou égale} revient à faire une comparaison du type
	\textit{supérieur}. On peut alors remplacer l’instruction de 
	cette comparaison par une autre.
}

\paragraph{}{
	Malheureusement, par manque de temps, la réalisation du processeur \textsc{Nono 2}
	reste à ce jour inachevée.
}