\documentclass[a4paper]{article}
\usepackage[utf8]{inputenc}
\usepackage[frenchb]{babel}
\usepackage[T1]{fontenc}
\usepackage{graphicx}
\usepackage{ifpdf}
\usepackage{hyperref}
\usepackage{amsmath}
\usepackage{listings}
\usepackage{mips}
\usepackage{slashbox}

\title{Processeu \textsc{Nono} 1et 2}
\author{Cédric \textsc{Bois} \and Benjamin \textsc{Sientzoff}}
\date{\today}
\ifpdf
\hypersetup{
    pdfauthor={Cédric Bois, Benjamin Sientzoff},
    pdftitle={Réalisation processeur Nono-1 et Nono-2},
}
\fi
\begin{document}

	% page de garde avec sommaire
	\maketitle
	\vspace{15cm}
	\par{Université de Nantes - Licence 3 Informatique}
	\par{\textsc{X5I0030} Architecture des Ordinateurs}
	\newpage
	% sommaire
	\tableofcontents
	\newpage % passer à la page suivante
	
	\section*{Introduction}

		\paragraph{}{
		Dans le cadre du cours intitulé \textit{Architecture des ordinateurs}, nous devons recréer
		un processeur Nono-1. Par la suite, ce processeur sera modifier pour devenir Nono-2. Ce 
		rapport retrace comment nous avons réalisé le premier processeur.
		}
		
		\paragraph{}{
		Les circuits électroniques présentés sont produits avec le logiciel \textit{Logisim}. Ces
		circuits et les différents fichiers permettant notamment de programmer le processeur sont 
		fournis avec la version numérique de ce rapport. Les images RAM peuvent être directement 
		chargées dans la RAM des processeurs Nono. Ces images correspondent aux programmes 
		compilés pour ces architectures et peuvent être exécutés directement dans \textit{Logisim}.
		}
		
		\paragraph{}{
		Nous commençons par présenter les différents sous-circuits composants le \textsc{Nono 1}.
		Puis nous détaillons son utilisation. Enfin, une brève partie explique comment 
		transformer \textsc{Nono 1} en \textsc{Nono 2}.
		}
	
	\newpage
	%\section{Réalisation de Nono-1}
	
		\section{Opcode des instructions}
			
			\paragraph{}{
			Nono-1 et Non-2 sont des processeurs utilisant l'assembleur MIPS. Les 
			instructions disponibles sur Nono-1 sont présentés au tableau de la figure
			\ref{tab_opcode}. On remarque que les instructions reconnues sont relativement
			restreintes. Ces instructions sont de trois formats différents comme on peut
			le voir à la figure \ref{format_inst}\footnote{Tiré du sujet du projet rédigé par M. Frédéric \textsc{Goualard}}.
			}
			
			\begin{figure}[!ht]
			\centering
			\includegraphics[scale=0.2]{formats_instructions.png}
			\caption{\label{format_inst} Formats des instructions}
			\end{figure}
			
				\subparagraph{Le Format F1}{
				Le format F1 est composé de quatre paquets de bits.
				Le premier est sur quatre bits, il correspond au code de l'instruction,
				et c'est le cas pour tous les formats d'instructions.
				Les trois paquets suivants, sur quatre bits. Ce format est utilisé typiquement
				pour des opérations faisant intervenir trois registres. Le premier correspond
				à la destination du résultat et les deux suivants aux registres contenant les
				opérantes.
				}
				\subparagraph{Le Format F2}{
				Le format F2 est composé de trois paquets de bits.
				Le premier est sur quatre bits, il correspond au code de l'instruction.
				Les 4 bits suivants correspondent à un nom de registre et les 8 derniers
				à une valeur immédiate. Ce format est typiquement utilisé pour l'instruction
				\textit{li}.
				}
				\subparagraph{Le Format F3}{
				Le format F3 peut être divisé en quatre parties. C'est le format utilisé
				pour les sauts. Les quatre bits correspondent à l'opcode de l'instruction.
				Le paquet des quatre bits et le suivant constitué des quatre autres bits
				suivant correspondent à des noms de registres. Enfin les derniers bits (au
				nombre de quatre) correspondent à un offset. Pour les sauts, cela correspond
				à l'adresse de l'étiquette où effectuer le saut. 
				}
			
			\begin{figure}
			\centering
			\begin{tabular}{|p{4cm}|c|c|c|c|}
				\hline Instruction et paramètres & Format & Opcode  \\ 
				\hline \texttt{add r$_{d}$, r$_{s}$, r$_{t}$} & F$_{1}$ & \texttt{1000} \\ 
				\hline \texttt{sub r$_{d}$, r$_{s}$, r$_{t}$} & F$_{1}$ & \texttt{1001}  \\ 
				\hline \texttt{or r$_{d}$, r$_{s}$, r$_{t}$} & F$_{1}$ & \texttt{1010}  \\ 
				\hline \texttt{and r$_{d}$, r$_{s}$, r$_{t}$} & F$_{1}$ & \texttt{1011}  \\ 
				\hline \texttt{not r$_{d}$, r$_{s}$} & F$_{1}$ & \texttt{1100} \\ 
				\hline \texttt{shl r$_{d}$, r$_{s}$, r$_{t}$} & F$_{1}$ & \texttt{1101} \\ 
				\hline \texttt{shr r$_{d}$, r$_{s}$, r$_{t}$} & F$_{1}$ & \texttt{1110} \\ 
				\hline \texttt{li r$_{d}$, $val$} & F$_{2}$ & \texttt{1111} \\ 
				\hline \texttt{halt} & F$_{1}$ & \texttt{0000} \\ 
				\hline \texttt{b \textit{offset}} & F$_{3}$ & \texttt{0001} \\
				\hline \texttt{beq r$_{s}$, r$_{t}$, \textit{offset}} & F$_{3}$ & \texttt{0010} \\ 
				\hline \texttt{bne r$_{s}$, r$_{t}$,\textit{offset}} & F$_{3}$ & \texttt{0011} \\ 
				\hline \texttt{bge r$_{s}$, r$_{t}$, \textit{offset}} & F$_{3}$ & \texttt{0100} \\ 
				\hline \texttt{ble r$_{s}$, r$_{t}$, \textit{offset}} & F$_{3}$ & \texttt{0101} \\ 
				\hline \texttt{bgt r$_{s}$, r$_{t}$, \textit{offset}} & F$_{3}$ & \texttt{0110} \\ 
				\hline \texttt{blt r$_{s}$, r$_{t}$, \textit{offset}} & F$_{3}$ & \texttt{0111} \\ 
				\hline 
				\end{tabular}
			\caption{
				\label{tab_opcode}
				\textit{Opcode} des différentes instruction du processeur \textsc{Nono 1}
			}
			\end{figure}
			
			\paragraph{}{
			Le choix des opcodes n'a pas était fait au hasard. En effet, en regardant le nombre
			d'instructions pour les sauts et le nombre d'opérations faisant appellent à l'unité
			arithmétique et logique du processeur, on s’aperçoit qu'ils peuvent être divisé 
			en deux 	groupes. On a donc regroupés les opcodes en trois groupes. Le premier 
			correspond aux opcodes qui commence par un $1$, ce sont les instructions qui font 
			appel à l'UAL. Le second groupe, les opcodes commencent par un $0$, correspondent 
			aux sauts. Enfin le dernier groupe est composé des autres instructions. Citons
			notamment les opcodes $0000$ et $1111$. Le tableau des instructions, leur format et
			les opcodes correspondants est présenté à la figure \ref{tab_opcode}.
			}
			
			\paragraph{}{
			Maintenant que nous avons définit nos opcodes, il est temps de concevoir les circuits
			électroniques composants le processeur \textsc{Nono 1}. Commençons par l'Unité 
			Arithmétique et Logique.
			}
	
		\section{L' unité arithmétique et logique}
			\paragraph{}{
	L'Unité Arithmétique et Logique, abrégé UAL, permet de faire des calculs
	basiques (additions, divisions, décallages de bits, etc.). Elle effectue
	les calcules sur huit bits. L'UAL a trois entrées. La première sur trois bits
	permet de préciser le code l'opérateur à	effectuer. Les deux autres entrées
	sur hiut bits sont les opérantes de l'opération demandée. \newline
	En sortie, sur \textit{ouput} on peut lire le résultat de l'opération. Il
	y a également quatres drapeaux comme détaillés ci-dessous.
}

\begin{itemize}
	\item[CF] pour \textit{Carrie Flag} est le drapeau levé lorsque que l'opération
	génère une retenue.
	\item[ZF] pour \textit{Zero Flag} qui est armé lorsque que le résultat comporte uniquement
	des $0$.
	\item[OF] pour \textit{Over Flow} qui est un drapeau levé lorsque on dépasse
	la capacité des nombres représentés.
	\item[SF] pour \textit{Sign Flag} qui est levé lorsque le résultat a son
	bit de poids fort à $1$, c'est un nombre signé.
\end{itemize}

\begin{figure}
	\centering
	\includegraphics[scale=0.4,origin=c]{circuits/UAL.png}
	\label{ual_circ}
	\caption{Sch\'{e}ma \'{e}lectronique de l'Unit\'{e} Arithm\'{e}tique et Logique}
\end{figure}

\begin{figure}
	\begin{center}
	\begin{tabular}{|c|c|c|c|c|}
		\hline
		\backslashbox{b3b2}{b1b0} & $0$ & $01$ & $11$ & $10$ \\ 
		\hline 
		$00$ & $0$ & $0$ & $0$ & $0$ \\ 
		\hline 
		$01$ & $0$ & $0$ & $0$ & $0$ \\ 
		\hline 
		$11$ & $1$ & $1$ & $0$ & $1$ \\ 
		\hline 
		$10$ & $1$ & $1$ & $1$ & $1$ \\ 
		\hline 
	\end{tabular} 
	\end{center}
	\label{karnaugh_ual}
	\caption{Tableau de Karnaugh pour le décodage de \textit{ctrlUAL}}
\end{figure}

\paragraph{}{
	Le schéma électronique de l'UAL est présenté à la figure \ref{ual_circ}.
	La partie qui décode le signal \textit{ctrlUAL} correspond au tableau de
	Karnaugh de la figure \ref{karnaugh_ual} duquel on extrait l'équation :
	\begin{equation}
		b3 . \neg(b2) + b3 . b2 \neg(b1) + b3 . b1 \neg(b0)
		\label{equation_ual}
	\end{equation}
	Cette équation nous permet alors de réaliser le circuit électronique 
	décodant \textit{ctrlAUL}.
}
			
		\section{Le contrôleur de saut}
			\paragraph{}{
	Le second circuit éléectronique composant le processeur est
	le contrôleur de saut. Son rôle est de mettre à jour \textit{PC}
	à jour en fonction du résultat de l'opération que vient d'effectuer
	l'UAL pour le cycle suivant.
	Le saut est déterminé en fonction des indicateurs que le circuits
	à en entrées, c'est-à-dire \textit{SF} et \textit{ZF}.
}

\begin{figure}[!ht]
	\centering
	\includegraphics[scale=0.4,origin=c]{circuits/control_saut.png}
	\label{control_saut_circ}
	\caption{Sch\'{e}ma \'{e}lectronique du contr\^{o}leur de sauts}
\end{figure}

\paragraph{}{
	Le schéma électronique du contrôleur de saut est à la figure
	\ref{control_saut_circ}. 
}
			
		\section{Le décodeur d'instructions}
			\paragraph{}{
	intro, explications
}

\begin{figure}
	\includegraphics[scale=0.6]{circuits/deco_instru.png}
	\label{selec_reg_circ}
	\caption{Sch\'{e}ma \'{e}lectronique pour le d\'{e}codeur d'instructions}
\end{figure}
			
		\section{La sélection des registres}
			\paragraph{}{
	Le sélecteur de registres permet de déterminer le registre
	dans lequel on va lire ou écrire un octet. Son circuit électronique
	est présenté à la figure \ref{selec_reg_circ}.
	Sont fonctionnement est relativement simple
}

\begin{figure}
	\centering
	\includegraphics[scale=0.8,origin=c]{circuits/selec_reg.png}
	\caption{
		\label{selec_reg_circ}
		Sch\'{e}ma \'{e}lectronique pour la s\'{e}lection de registres
	}
\end{figure}
			
		\section{Le banc de registres}
			\paragraph{}{
	Le banc de registres constitue la mémoire du processeur.
	Il est composé 16 registres d'un octet. Un tel circuit
	est composé de bascules D en cascade. Il y en a une pour
	chaque registre.
	L'écriture d'un registre est possible uniquement lorsque
	\textit{regWrite} est à vrai.
}

\begin{figure}
	\centering
	\includegraphics[scale=0.3,angle=90,origin=c]{circuits/banc_reg.png}
	\label{banc_reg_circ}
	\caption{Sch\'{e}ma \'{e}lectronique pour le banc de registres}
\end{figure}

\begin{figure}
	\centering
	\includegraphics[scale=0.3,origin=c]{circuits/banc_reg_selec.png}
	\caption{
		\label{banc_reg_selec_circ}
		Sch\'{e}ma \'{e}lectronique pour le sélecteur du banc de registres
	}
\end{figure}
			
		\paragraph{}{
				Trouvez à la figure \ref{nono_circ} le schéma global du processeur \textsc{Nono 1}
				utilisant les circuits présentés dans ce rapport.
		}
		
		\begin{figure}
			\centering
			\includegraphics[scale=0.35]{circuits/Nono-1.png}
			\caption{
			\label{nono_circ}
			Schéma de \text{Nono 1}
			}
		\end{figure}
					
	
	%\section{Processeurs Nono-1 et Nono-2}
		\section{Nono-1}
			\paragraph{PGCD}{
	On commence par utiliser le processeur en implémentant la fonction qui
	calcule le plus grand diviseur commun de deux nombres.
	Pour ce faire, nous traduisons d'abord le code C écrit dans le sujet
	(figure \ref{pgcd_c}), puis, on traduit le code MIPS (figure 
	\ref{pgcd_asm}) en hexadécimale (figure \ref{pgcd_hexa}), en se 
	basant sur les opcodes définis plus tôt et le format des instructions.
}

\begin{figure}
	\lstset{
		frame=single,
		numbers=left,
		numbersep=5pt,
		language=C++
	}
	\begin{lstlisting}
int i = 27;
int j = 24;
while (i != j)
{
	if (i > j)
	{
		i -= j;
	}
 	else
 	{
		j -= i;
	}
}
	\end{lstlisting}
	\caption{
		\label{pgcd_c}
		Code C de la fonction \textit{pgcd} tirée du sujet
	}
\end{figure}

\begin{figure}
	\lstset{
		frame=single,
		numbers=left,
		numbersep=5pt,
		language=[mips]Assembler,
		morekeywords={halt,shr}
	}
	\begin{lstlisting}
	.data
	.text
	.globl __start
	
__start:
	li $t1, 27	#i
	li $t0, 24	#j
	
while :
	beq $t1, $t0, end_while
if:
	ble $t1, $t0, else
then:
	sub $t1, $t1, $t0
	b end_if
else:
	sub $t0, $t0, $t1
end_if:
	b while
end_while:

# fin du programme
halt

	\end{lstlisting}
	\caption{
		\label{pgcd_asm}
		Code MIPS du programme \textit{pgcd}
	}
\end{figure}

\begin{figure}
	\lstset{
		frame=single,
		numbers=left,
		numbersep=5pt,
		language=[mips]Assembler
	}
	\begin{lstlisting}
0xf11b # li $t0 24
0xf018 # li $t1 27
0x2105 # bed $t1 $t0 5
0x5102 # ble $t1 $t0 2
0x9110 # sub $t1 $t1 $t0
0x1001 # b 1
0x9001 # sub $t0 $t0 $t1
0x100a # b -6
0x0000 # halt
	\end{lstlisting}
	\caption{
		\label{pgcd_hexa}
		Code hexadécimal du programme \textit{pgcd} compilé pour \textsc{Nono 1}
	}
\end{figure}

\paragraph{}{
	Ce programme nous donne bien le résultat attendu dans les registres : $3$.
}

\paragraph{NTZ}{
	Pour aller plus loin nous avons codé un second programme. Nous avons choisi
	d'implémenter la fonction \textit{ntz} présente sur l'examen de 2012/2013 de cette
	même matière. \newline
	À la figure \ref{ntz_c} on a le code C de la fonction, à la figure \ref{ntz_asm}
	son code MIPS et à la figure \ref{ntz_asm} le code hexadécimal du programme
}

\begin{figure}
	\lstset{
		frame=single,
		numbers=left,
		numbersep=5pt,
		language=C++
	}
	\begin{lstlisting}
int ntz( unsigned int x )
{
	int n = 0;
	x = ~x & (x-1);
	while (x != 0)
	{
		n = n + 1;
		x = x >> 1;
	}
	return n;
}
	\end{lstlisting}
	\caption{
		\label{ntz_c}
		Code C de la fonction \textit{ntz} tirée du sujet de l’examen de 2012-2013
	}
\end{figure}

\begin{figure}
	\lstset{
		frame=single,
		numbers=left,
		numbersep=5pt,
		language=[mips]Assembler,
		morekeywords={halt,shr}
	}
	\begin{lstlisting}
	.data
	.text
	.globl __start
	
__start:
# initialisation des variables
	li $t0 32		# x = 32
	li $t1 0		# n = 0
	li $t3 1		# pour -1
	li $t4 0		# pour zero
		
	sub $t2 $t0 $t3		# x - 1
	not $t0 $t0		# not x
	and $t0 $t0 $t2		# and(x-1)
	
while:
	beq $t0, $t4, end_while	
	add $t1, $t1, $t3	# n = n + 1
	shr $t0, $t0, 1		# x = x >> 1
	b while			
end_while:
	# fin du programme
	halt
	\end{lstlisting}
	\caption{
		\label{ntz_asm}
		Code MIPS de la fonction \textit{ntz}
	}
\end{figure}

\begin{figure}
	\lstset{
		basicstyle=\ttfamily,
		frame=single,
		numbers=left,
		numbersep=5pt,
		language=[mips]Assembler
	}
	\begin{lstlisting}
0xf020 # li $t2 32
0xf100 # li $t1 0
0xf301 # li $t3 1
0xf400 # li $t4 0
0x9203 # sub $t2 $t0 $t3
0xc000 # not $t0 $t0
0xb002 # and $t0 $t0 $t2
0x2043 # beq $t0 $t4 3
0x8113 # add $t1 $t1 $t3
0xe003 # shr $t0 $t0 1
0x100c # b -4
0x0000 # halt
	\end{lstlisting}
	\caption{
		\label{ntz_hexa}
		Code hexadécimal de la fonction \textit{ntz}
	}
\end{figure}

\paragraph{}{
	Le programme \textit{ntz} stock son résultat dans le second registre.
}

\paragraph{HALT!}{
	D'après le fonctionnement général de \textsc{Nono 1}, l’instruction \textit{halt} donne l'adresse
	mémoire de la dernière instruction. Le problème est que le processeur ne 
	s’arrête pas lorsqu'il a atteint cette instruction. et il va boucler sur cette 
	instruction. Ainsi, si cette dernière est utilisée, elle va être exécutée à l'infini. 
}
		\section{Nono-2}
			\paragraph{}{
	Un processeur \textsc{Nono 2} est un processeur \textsc{Nono 1}
	implémentant des routines. La gestion des routines suppose que
	le processeur est capable de gérer plus choses.	
}

	\subparagraph{Sauts longs}{
	Un appel à une fonction est traduit grossièrement en assembleur par 
	un saut dans le code du programme.
	Les sauts longs (les instructions de type \verb|j étiquette|)
	ne sont pas du même type que les sauts effectués par un \verb|b|.
	Pour cela, il faut être capable d'incrémenter le pointeur d'instruction
	\textit{PC}. Ce qui est déjà possible avec \textsc{Nono 1}.
	}
	
	\subparagraph{Paramètres de fonctions et pile}{
	Lorsqu'on appelle une fonction, en général, c'est pour effectuer un calcul. 
	Calcul qu'on souhaite effectuer sur des paramètres spécifiques. Il faut donc
	être capable de passer ces paramètres à une fonction mais également récupérer
	son résultat. Lorsqu'on utilise une fonction 	\textit{simple} (prenant en 
	paramètres des scalaires), il suffit d'initialiser
	des registres spécifiques pour que la fonction puisse récupérer les valeurs
	données. Puis la fonction modifie certains registres lorsqu'elle renvoie un résultat.
	Or, \textsc{Nono 1} ne permet pas cela car il n'a pas de registres dédiés. \newline
	D'autre part, lorsque la fonction prend en paramètre des types plus complexes
	(un tableau par exemple) ou qu'elle appelle elle-même une fonction, il faut
	alors gérer les cadres de pile des fonctions. Ce qui n'est pas possible actuellement.
	}
	
\paragraph{}{
	Tous ces éléments suggèrent également l'utilisation d'instructions que le processeur
	ne possède pas. Par exemple, les instructions de saut, les chargements en mémoire 
	(gestion de la pile), etc. Pour cela, il faudrait, par exemple, étendre le jeu
	d'instructions actuel. Cela entraînerai alors une modification importante du 
	processeur, notamment car la taille des opcodes ne serait plus la même. \newline
	Une seconde solution envisageable et réalisable, serait de sacrifier
	des instructions du jeu actuel au profil de nouvelles. Par exemple, une comparaison
	de type \textit{inférieur ou égale} revient à faire une comparaison du type
	\textit{supérieur}. On peut alors remplacer l’instruction de 
	cette comparaison par une autre.
}

\paragraph{}{
	Malheureusement, par manque de temps, la réalisation du processeur \textsc{Nono 2}
	reste à ce jour inachevée.
}
	

	
	
	\newpage
	\section*{Conclusion}
		\paragraph{}{
		La réalisation d'un processeur n'est pas trivial. Un processeur est un 
		composant relativement complexe. Mais la division de ce dernier par unités
		spécialisées permet de simplifier sa conception. \newline
		}
		
		\paragraph{}{
		En réalisant nous même un processeur, on a put prendre conscience à 
		quel point un langage d'assemblage est lié à une architecture physique.
		}
		
	\newpage
	\listoffigures
		
\end{document}
